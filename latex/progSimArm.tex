% !TeX spellcheck = sk
\documentclass{article}

\usepackage{tikz}
\usetikzlibrary{arrows}
\usepackage{verbatim}
\usepackage{xcolor}
\usepackage{listings}
\usepackage{graphicx}
\usepackage[slovak]{babel}
\usepackage[latin2,utf8x]{inputenc}
\usetikzlibrary{shapes,arrows}
% Define block styles
\tikzstyle{decision} = [diamond, draw, fill=blue!20, 
    text width=4.5em, text badly centered, node distance=3cm, inner sep=0pt]
\tikzstyle{block} = [rectangle, draw, fill=blue!20, 
    text width=10em, text centered, rounded corners, minimum height=4em]
\tikzstyle{line} = [draw, -latex']
\tikzstyle{cloud} = [draw, ellipse,fill=red!20, node distance=3cm,
    minimum height=2em]

\title{Programátorská príručka pre SimArm}
\date{December 06 2016}
\author{Juraj Fojtí­k}
\begin{document}
\maketitle
\section{Stromový diagram:}
\tikzstyle{int}=[draw, fill=blue!20, minimum size=2em]
\tikzstyle{init} = [pin edge={to-,thin,black}]
\begin{tikzpicture}[node distance=2.5cm,auto,>=latex']
    \node [int] (s) {Server};
    \node [int] (3) [below of=s] {Klient 3};
    \node [int] (2) [left of=3] {Klient 2};
    \node [int] (1) [left of=2] {Klient 1};
    \node [int] (4) [right of=3] {Klient 4};
    \node [int] (5) [right of=4] {Klient 5};
    \path[<->] (s) edge node {} (1);
    \path[<->] (s) edge node {} (2);
    \path[<->] (s) edge node {} (3);
    \path[<->] (s) edge node {} (4);
    \path[<->] (s) edge node {} (5);
\end{tikzpicture}
\newline\newline
Server: prenos dát medzi klientami\newline
Klient1: Riadenie prvého motora(v čase)\newline
Klient2: Riadenie druhého motora(v čase)\newline
Klient3: Nastavovanie základných parametrov robotickej ruky(užívateľské rozhranie)\newline
Klient4: Prepočet pozícií pre grafické zobrazenie robotickej ruky\newline
Klient5: Zobrazenie robotickej ruky v okne(užívateľské rozhranie) \newline\newline
\section{Server}
- dekódovanie argumentov(-i,-p,-h)\newline
- vytvorenie soket deskriptoru (socket)\newline
- priradenie parametrov nastavenia soketu do štruktúry (AFINET práca na lokálnej sieti,port,INADDRANY prijímanie všetkých ip adries)\newline
- priradenie informácii zo štruktúry do soket deskriptoru(bind)\newline
- nastavenie počúvania serveru(listen)\newline
- vytvorenie zdielanej pamäte(vytvorZP) \newline
- pri mapovanie zdielanej pamäte(pripojZP) \newline
- inicializácia dát štruktúry(initRobotArm)\newline
- vytvorenie semaforu (semCreate)\newline
- inicializácia semaforu(semInit)\newline
- čakanie na klienta(accept)\newline
- po pripojení klienta priradenie soket deskriptoru, vytvorenie nového procesu+test správnosti vytvorenia procesu(fork)\newline
- v procese serveru - uzavretie soket deskriptoru klienta, vrátenie sa na čakanie klienta\newline
- v procese klienta - uzavretie soket deskriptoru serveru\newline
- nastavenie signálov pre zrušenie soketu a ctrl+c - udalosť(signal,SIGINT,SIGPIPE)\newline
- prijatie id klienta(recv)\newline
- pripojenie zdielanej pamäte(pripojZP)\newline
- prijímanie a odosielanie dát závisí na tom, ktoré id bolo prijaté
\section{Klient č.1}
- dekódovanie argumentov(-i,-p,-h)\newline
- pripojenie sa na server(pripoj)\newline
- nastavenie signálov pre zrušenie soketu a ctrl+c - udalosť(signal,SIGINT,SIGPIPE)\newline
- odoslanie id klienta(odosliInt)\newline
- vytvorenie vlákna pre čítanie zo soketu(pthread\_create)\newline
- práca s premennou pdataArm->actAlfa\newline
- odoslanie pdataArm->actAlfa\newline
\section{Klient č.2}
- dekódovanie argumentov(-i,-p,-h)\newline
- pripojenie sa na server(pripoj)\newline
- nastavenie signálov pre zrušenie soketu a ctrl+c - udalosť(signal,SIGINT,SIGPIPE)\newline
- odoslanie id klienta(odosliInt)\newline
- vytvorenie vlákna pre čítanie zo soketu(pthread\_create)\newline
- práca s premennou pdataArm->actBeta\newline
- odoslanie pdataArm->actBeta\newline
\section{Klient č.3}
- dekódovanie argumentov(-i,-p,-h)\newline
- pripojenie sa na server(pripoj)\newline
- nastavenie signálov pre zrušenie soketu a ctrl+c - udalosť(signal,SIGINT,SIGPIPE)\newline
- odoslanie id klienta(odosliInt)\newline
- nastavenie funkcie pre signál(funkciaCasovac) \newline
- vytvorenie časovača so signálom(vytvorCasovac)\newline
- nastavenie času pre časovač(opakovanyCasovac)\newline
- v časovači odosielanie, prijímanie dát štruktúry vymazanie plochy a výpis\newline
- v hlavnom vlákne načítavanie z klávesnice\newline
\section{Klient č.4}
- dekódovanie argumentov(-i,-p,-h)\newline
- pripojenie sa na server(pripoj)
- nastavenie signálov pre zrušenie soketu a ctrl+c - udalosť(signal,SIGINT,SIGPIPE)\newline
- odoslanie id klienta(odosliInt)\newline
- načítanie dát zo soketu(nacitajRobotArm)\newline
- vypočítanie aktualneho stavu bodov a uhlov(angleArmAct)\newline
- odoslanie dát cez soket(send)\newline
\section{Klient č.5}
- dekódovanie argumentov(-i,-p,-h)\newline
- pripojenie sa na server(pripoj)
- nastavenie signálov pre zrušenie soketu a ctrl+c - udalosť(signal,SIGINT,SIGPIPE)\newline
- odoslanie id klienta(odosliInt)\newline
- vytvorenie vlákna pre načítavanie zo soketu(pthread\_create)\newline
- vytvorenie okna(cvNamedWindow)\newline
- nastavenie callbacku pre myš(cvSetMouseCallback)\newline
- vyplotovanie bodov\newline
\end{document}